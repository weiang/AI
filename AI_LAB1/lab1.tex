\documentclass[4paper,10pt]{paper}
\usepackage{listings}
\usepackage{hyperref}
\usepackage{amsmath,amssymb,amsthm}
\usepackage{xcolor}
\usepackage{xeCJK}
%\usepackage{amsmath}
\usepackage{algorithm}
\usepackage{algorithmicx}
\setCJKmainfont{AR PL UKai CN}

\lstset{numbers=none,
  numberstyle=\scriptsize,
  flexiblecolumns=false,
  language=Haskell,
  frame=shadowbox,
  basicstyle=\ttfamily\small,
  breaklines=true,
  extendedchars=true,
  escapechar=\%,
  texcl=true,
  showstringspaces=false,
  keywordstyle=\bfseries,
  tabsize=4}
  
\title{人工智能 \\ 
	\center{实验一  $N$皇后问题}}
\author{昂伟 PB11011058}
\date{ \today }

\begin{document}

\maketitle
\section*{算法分析}
	\noindent 输入: N \\
	输出: 三组可行解 \\
	$size\_t\  **NQueen(int\ N)$根据$N$的不同调用三个不同的子函数来解决N皇后问题以达到最好的运行效果,降低时间和空间复杂度.

	\begin{description}
	\item[$0< N < 15$] 调用$void\ nqueen1(int\ col, int N)$.
	\item[$15 <= N <= 3000$] 调用$void\ nqueen2(size\_t *queen, int\ N)$.
	\item[$3000 < N$] 调用$void\ nqueen3(size\_t *queen, int\ N)$.
	\end{description}
	
\section*{时空复杂度分析}
	\begin{description}
		\item $void\ nqueen1(int col, int N)$是一个递归函数,时间复杂度递推关系为:$$T_n = n * T_{n - 1}$$
		所以,$T_n = O(N!)$.因为是递归函数,所以需要维护一个栈,即空间复杂度为$S_n = O(N)$.
		
		\item $void\ nqueen2(size_t\ *queen, int\ N)$的算法描述为:
		\begin{lstlisting}
	begin 
		while (解有冲突) {
			(1) 产生一个随机解s
			(2) for all i, j: where 皇后s[i]或皇后s[j]有冲突:
				(2.1) if (交换s[i],s[j]能减少冲突)
						then 交换s[i], s[j]
		}
	end 
		\end{lstlisting}
		\paragraph{分析}{算法步骤(1)产生随机解的时间复杂度为$O(N)$,步骤(2)为一个双层for循环,所以时间复杂度为$O(N^2)$.
		而最外层的while循环的最坏循环次数为$O(N)$.所以本算法的最坏时间复杂度为$O(N^3)$.
		由于需要维护两个大小为$2 * N - 1$的数组,所以空间复杂度为$S_N = O(N)$.}
		
		\item $void\ nqueen3(size_t\ *queen, int\ N)$的时间复杂度为:$T_N = O(n)$.空间复杂度为$S_N = O(N)$.具体分析参见\href{http://dl.acm.org/citation.cfm?id=101343}{http://dl.acm.org/citation.cfm?id=101343}.
	\end{description}
	
	\subsection*{总结}
	当$N$很小时($N<15$),使用递归算法可以保证相当好的运行效率;当$N$特别大时($N > 3000$),采用局部搜索算法有非常好的运行效率(可在数分钟内解决千万皇后问题);当$15<N<3000$时,采用类爬山算法可以获得较好的运行效率.
	
	总的来说,本程序可以解决各种规模的N皇后问题.
\end{document}